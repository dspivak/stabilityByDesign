\documentclass[11pt, one side, article]{memoir}


\settrims{0pt}{0pt} % page and stock same size
\settypeblocksize{*}{35.3pc}{*} % {height}{width}{ratio}
\setlrmargins{*}{*}{1} % {spine}{edge}{ratio}
\setulmarginsandblock{.98in}{.98in}{*} % height of typeblock computed
\setheadfoot{\onelineskip}{2\onelineskip} % {headheight}{footskip}
\setheaderspaces{*}{1.5\onelineskip}{*} % {headdrop}{headsep}{ratio}
\checkandfixthelayout


\usepackage{amsthm}
\usepackage{mathtools}

\usepackage[inline]{enumitem}
\usepackage{ifthen}
\usepackage[utf8]{inputenc} %allows non-ascii in bib file
\usepackage{xcolor}

\usepackage[backend=biber, backref=true, maxbibnames = 10, style = alphabetic]{biblatex}
\usepackage[bookmarks=true, colorlinks=true, linkcolor=blue!50!black,
citecolor=orange!50!black, urlcolor=orange!50!black, pdfencoding=unicode]{hyperref}
\usepackage[capitalize]{cleveref}

\usepackage{tikz}

\usepackage{amssymb}
\usepackage{newpxtext}
\usepackage[varg,bigdelims]{newpxmath}
\usepackage{mathrsfs}
\usepackage{dutchcal}
\usepackage{fontawesome}
\usepackage{ebproof}
\usepackage{stmaryrd}
\usepackage{float}

% cleveref %
  \newcommand{\creflastconjunction}{, and\nobreakspace} % serial comma
  \crefformat{enumi}{\card#2#1#3}
  \crefalias{chapter}{section}


% biblatex %
  \addbibresource{Library20231216.bib} 


% hyperref %
  \hypersetup{final}

% enumitem %
  \setlist{nosep}
  \setlistdepth{6}



% tikz %



  \usetikzlibrary{ 
  	cd,
  	math,
  	decorations.markings,
		decorations.pathreplacing,
  	positioning,
  	arrows.meta,
  	shapes,
		shadows,
		shadings,
  	calc,
  	fit,
  	quotes,
  	intersections,
    circuits,
    circuits.ee.IEC
  }
  

\tikzset{
	tick/.style={postaction={
  	decorate,
    decoration={markings, mark=at position 0.5 with
    	{\draw[-] (0,.4ex) -- (0,-.4ex);}}}
  }
} 
\tikzset{
	slash/.style={postaction={
  	decorate,
    decoration={markings, mark=at position 0.5 with
    	{\node[font=\footnotesize] {\rotatebox{90}{$\sim$}};}}}
  }
} 

% Adjunctions
\newcommand{\adj}[5][30pt]{%[size] Cat L, Left, Right, Cat R.
\begin{tikzcd}[ampersand replacement=\&, column sep=#1]
  #2\ar[r, shift left=4.5pt, "#3"]
  \ar[r, phantom, "\scriptstyle\Rightarrow"]\&
  #5\ar[l, shift left=4.5pt, "#4"]
\end{tikzcd}
}

\newcommand{\adjr}[5][30pt]{%[size] Cat R, Right, Left, Cat L.
\begin{tikzcd}[ampersand replacement=\&, column sep=#1]
  #2\ar[r, shift left=7pt, "#3"]\&
  #5\ar[l, shift left=7pt, "#4"]
  \ar[l, phantom, "\scriptstyle\Leftarrow"]
\end{tikzcd}
}

\newcommand{\adjpos}[6][30pt]{%[size] Cat L, Left, Right, Cat R.
\begin{tikzcd}[ampersand replacement=\&, column sep=#1]
  #2\ar[r, shift left=4.5pt, "#3"]
  \ar[r, phantom, "\scriptstyle#6"]\&
  #5\ar[l, shift left=4.5pt, "#4"]
\end{tikzcd}
}

\newcommand{\adjrpos}[6][30pt]{%[size] Cat R, Right, Left, Cat L.
\begin{tikzcd}[ampersand replacement=\&, column sep=#1]
  #2\ar[r, shift left=4.5pt, "#3"]\&
  #5\ar[l, shift left=4.5pt, "#4"]
  \ar[l, phantom, "\scriptstyle#6"]
\end{tikzcd}
}

\newcommand{\xtickar}[1]{\begin{tikzcd}[baseline=-0.5ex,cramped,sep=small,ampersand 
replacement=\&]{}\ar[r,tick, "{#1}"]\&{}\end{tikzcd}}

\newcommand{\xslashar}[1]{\begin{tikzcd}[baseline=-0.5ex,cramped,sep=small,ampersand 
replacement=\&]{}\ar[r,slash, "{#1}"]\&{}\end{tikzcd}}

 
  % amsthm %
\theoremstyle{definition}
\newtheorem{definitionx}{Definition}[chapter]
\newtheorem{examplex}[definitionx]{Example}
\newtheorem{remarkx}[definitionx]{Remark}
\newtheorem{notation}[definitionx]{Notation}


\theoremstyle{plain}

\newtheorem{theorem}[definitionx]{Theorem}
\newtheorem{proposition}[definitionx]{Proposition}
\newtheorem{corollary}[definitionx]{Corollary}
\newtheorem{lemma}[definitionx]{Lemma}
\newtheorem{warning}[definitionx]{Warning}
\newtheorem{conjecture}[definitionx]{Conjecture}
\newtheorem*{theorem*}{Theorem}
\newtheorem*{proposition*}{Proposition}
\newtheorem*{corollary*}{Corollary}
\newtheorem*{lemma*}{Lemma}
\newtheorem*{warning*}{Warning}
%\theoremstyle{definition}
%\newtheorem{definition}[theorem]{Definition}
%\newtheorem{construction}[theorem]{Construction}

\newenvironment{example}
  {\pushQED{\qed}\renewcommand{\qedsymbol}{$\lozenge$}\examplex}
  {\popQED\endexamplex}
  
 \newenvironment{remark}
  {\pushQED{\qed}\renewcommand{\qedsymbol}{$\lozenge$}\remarkx}
  {\popQED\endremarkx}
  
  \newenvironment{definition}
  {\pushQED{\qed}\renewcommand{\qedsymbol}{$\lozenge$}\definitionx}
  {\popQED\enddefinitionx} 

    
%-------- Single symbols --------%
	
\DeclareSymbolFont{stmry}{U}{stmry}{m}{n}
\DeclareMathSymbol\fatsemi\mathop{stmry}{"23}

\DeclareFontFamily{U}{mathx}{\hyphenchar\font45}
\DeclareFontShape{U}{mathx}{m}{n}{
      <5> <6> <7> <8> <9> <10>
      <10.95> <12> <14.4> <17.28> <20.74> <24.88>
      mathx10
      }{}
\DeclareSymbolFont{mathx}{U}{mathx}{m}{n}
\DeclareFontSubstitution{U}{mathx}{m}{n}
\DeclareMathAccent{\widecheck}{0}{mathx}{"71}

%-------- Tikz diagrams --------%

\tikzset{
	oriented WD/.style={%everything after equals replaces "oriented WD" in key.
		every to/.style={out=0,in=180,draw},
    label/.style={
    	font=\everymath\expandafter{\the\everymath\scriptstyle},
      inner sep=0pt,
      node distance=2pt and -2pt},
    semithick,
    node distance=1 and 1,
    decoration={markings, mark=at position \stringdecpos with \stringdec},
    ar/.style={postaction={decorate}},
    execute at begin picture={\tikzset{
    	x=\bbx, y=\bby,
      every fit/.style={inner xsep=\bbx, inner ysep=\bby}}}
    },
    string decoration/.store in=\stringdec,
    string decoration={\arrow{stealth};},
    string decoration pos/.store in=\stringdecpos,
    string decoration pos=.7,
    bbx/.store in=\bbx,
    bbx = 1.5cm,
    bby/.store in=\bby,
    bby = 1.5ex,
    bb port sep/.store in=\bbportsep,
    bb port sep=1.5,
    % bb wire sep/.store in=\bbwiresep,
    % bb wire sep=1.75ex,
    bb port length/.store in=\bbportlen,
    bb port length=4pt,
    bb penetrate/.store in=\bbpenetrate,
    bb penetrate=0,
    bb min width/.store in=\bbminwidth,
    bb min width=1cm,
    bb rounded corners/.store in=\bbcorners,
    bb rounded corners=2pt,
    bb spider/.style={
    	bb port sep=1, bb port length=10pt, bbx=.4cm, bb min width=.4cm, bby=.8ex},
    bb small/.style={
    	bb port sep=1, bb port length=2.5pt, bbx=.4cm, bb min width=.4cm, bby=.7ex},
		bb medium/.style={
			bb port sep=1, bb port length=2.5pt, bbx=.4cm, bb min width=.4cm, bby=.9ex},
    bb/.code 2 args={%When you see this key, run the code below:
    	\pgfmathsetlengthmacro{\bbheight}{\bbportsep * (max(#1,#2)+1) * \bby}
      \pgfkeysalso{draw,minimum height=\bbheight,minimum
       width=\bbminwidth,outer sep=0pt,
         rounded corners=\bbcorners,thick,
         prefix after command={\pgfextra{\let\fixname\tikzlastnode}},
         append after command={\pgfextra{\draw
            \ifnum #1=0{} \else foreach \i in {1,...,#1} {
            	($(\fixname.north west)!{\i/(#1+1)}!(\fixname.south west)$) +(-\bbportlen,0) coordinate (\fixname_in\i) -- +(\bbpenetrate,0) coordinate (\fixname_in\i')}\fi 
  					%Define the endpoints of tickmarks
            \ifnum #2=0{} \else foreach \i in {1,...,#2} {
            	($(\fixname.north east)!{\i/(#2+1)}!(\fixname.south east)$) +(-
\bbpenetrate,0) coordinate (\fixname_out\i') -- +(\bbportlen,0) coordinate (\fixname_out\i)}\fi;
           }}}
		},
			bb name/.style={
     	append after command={
				\pgfextra{\node[anchor=north] at (\fixname.north) {#1};}
			}
		}
  }


%-------- Renewed commands --------%

\renewcommand{\ss}{\subseteq}

%-------- Other Macros --------%


\DeclarePairedDelimiter{\present}{\langle}{\rangle}
\DeclarePairedDelimiter{\copair}{[}{]}
\DeclarePairedDelimiter{\floor}{\lfloor}{\rfloor}
\DeclarePairedDelimiter{\ceil}{\lceil}{\rceil}
\DeclarePairedDelimiter{\corners}{\ulcorner}{\urcorner}
\DeclarePairedDelimiter{\ihom}{[}{]}

\DeclareMathOperator{\Hom}{Hom}
\DeclareMathOperator{\Mor}{Mor}
\DeclareMathOperator{\dom}{dom}
\DeclareMathOperator{\cod}{cod}
\DeclareMathOperator{\idy}{idy}
\DeclareMathOperator{\comp}{com}
\DeclareMathOperator*{\colim}{colim}
\DeclareMathOperator{\im}{im}
\DeclareMathOperator{\ob}{Ob}
\DeclareMathOperator{\Tr}{Tr}
\DeclareMathOperator{\el}{El}
\DeclareMathOperator{\Funn}{Fun}


\newcommand{\tw}{\Cat{Tw}}

\newcommand{\Set}[1]{\mathsf{#1}}%a named set
\newcommand{\ord}[1]{\mathsf{#1}}%an ordinal
\newcommand{\cat}[1]{\mathcal{#1}}%a generic category
\newcommand{\Cat}[1]{\mathbf{#1}}%a named category
\newcommand{\fun}[1]{\mathrm{#1}}%a function
\newcommand{\Fun}[1]{\mathit{#1}}%a named functor




\newcommand{\id}{\mathrm{id}}
\newcommand{\then}{\mathbin{\fatsemi}}

\newcommand{\cocolon}{:\!}


\newcommand{\iso}{\cong}
\newcommand{\too}{\longrightarrow}
\newcommand{\tto}{\rightrightarrows}
\newcommand{\To}[2][]{\xrightarrow[#1]{#2}}
\renewcommand{\Mapsto}[1]{\xmapsto{#1}}
\newcommand{\Tto}[3][13pt]{\begin{tikzcd}[sep=#1, cramped, ampersand replacement=\&, text height=1ex, text depth=.3ex]\ar[r, shift left=2pt, "#2"]\ar[r, shift right=2pt, "#3"']\&{}\end{tikzcd}}
\newcommand{\Too}[1]{\xrightarrow{\;\;#1\;\;}}
\newcommand{\from}{\leftarrow}
\newcommand{\fromm}{\longleftarrow}
\newcommand{\ffrom}{\leftleftarrows}
\newcommand{\From}[1]{\xleftarrow{#1}}
\newcommand{\Fromm}[1]{\xleftarrow{\;\;#1\;\;}}
\newcommand{\surj}{\twoheadrightarrow}
\newcommand{\inj}{\hookrightarrow}
\newcommand{\wavyto}{\rightsquigarrow}
\newcommand{\lollipop}{\multimap}
\newcommand{\imp}{\Rightarrow}
\newcommand{\down}{\mathbin{\downarrow}}
\newcommand{\fromto}{\leftrightarrows}
\newcommand{\tickar}{\xtickar{}}
\newcommand{\slashar}{\xslashar{}}
\newcommand{\card}{\,^{\#}}

\newcommand{\rdag}{^{\rotatebox{0}{$\dagger$}}}
\newcommand{\ldag}{^{\rotatebox{180}{$\dagger$}}}

\newcommand{\inv}{^{-1}}
\newcommand{\op}{^\tn{op}}

\newcommand{\tn}[1]{\textnormal{#1}}
\newcommand{\ol}[1]{\overline{#1}}
\newcommand{\ul}[1]{\underline{#1}}
\newcommand{\wt}[1]{\widetilde{#1}}
\newcommand{\wh}[1]{\widehat{#1}}
\newcommand{\wc}[1]{\widecheck{#1}}
\newcommand{\ubar}[1]{\underaccent{\bar}{#1}}



\newcommand{\bb}{\mathbb{B}}
\newcommand{\cc}{\mathbb{C}}
\newcommand{\nn}{\mathbb{N}}
\newcommand{\pp}{\mathbb{P}}
\newcommand{\qq}{\mathbb{Q}}
\newcommand{\zz}{\mathbb{Z}}
\newcommand{\rr}{\mathbb{R}}


\newcommand{\finset}{\Cat{Fin}}
\newcommand{\smset}{\Cat{Set}}
\newcommand{\lgset}{\Cat{SET}}
\newcommand{\smcat}{\Cat{Cat}}
\newcommand{\lgcat}{\Cat{CAT}}
\newcommand{\prof}{\mathbb{P}\Cat{rof}}
\newcommand{\catsharp}{\Cat{Cat}^{\sharp}}
\newcommand{\ppolyfun}{\mathbb{P}\Cat{olyFun}}
\newcommand{\ccatsharp}{\mathbb{C}\Cat{at}^{\sharp}}
\newcommand{\ccatsharpdisc}{\mathbb{C}\Cat{at}^{\sharp}_{\tn{disc}}}
\newcommand{\ccatsharplin}{\mathbb{C}\Cat{at}^{\sharp}_{\tn{lin}}}
\newcommand{\ccatsharpdisccon}{\mathbb{C}\Cat{at}^{\sharp}_{\tn{disc,con}}}
\newcommand{\sspan}{\mathbb{S}\Cat{pan}}
\newcommand{\en}{\Cat{End}}

\newcommand{\set}{\tn{-}\Cat{Set}}
\newcommand{\coalg}{\tn{-}\Cat{Coalg}}
\newcommand{\const}[1][\blank]{#1 !}

\newcommand{\true}{\texttt{true}}
\newcommand{\false}{\texttt{false}}


\newcommand{\blank}[1][1pt]{\hspace{#1}\cdot\hspace{#1}}
\newcommand{\rest}[2][\blank]{#1\big|_{#2}}



\newcommand{\slogan}[1]{\begin{center}\textit{#1}\end{center}}

\newcommand{\hh}[2][]{#1 \tn{\textit{#2}} #1}
\newcommand{\qqand}{\hh[\qquad]{and}}
\newcommand{\qand}{\hh[\quad]{and}}
\newcommand{\qqor}{\hh[\qquad]{or}}
\newcommand{\qor}{\hh[\quad]{or}}
\renewcommand{\iff}[1][\;\;]{#1\Leftrightarrow#1}
\newcommand{\ifff}[1][\;\;]{#1\xLeftrightarrow{\quad}#1}
\newcommand{\hi}[4][]{#1 #2 \tn{\textit{#4}} #3}
\newcommand{\where}[1][,]{\hi[#1]{\qquad}{\quad}{where}}
\newcommand{\qimplies}{\hh[\quad]{$\implies$}}

\newcommand{\dnote}[1]{{\color{blue}David says:}~#1\quad{\color{blue}$\lozenge$}}


% David extras




\newcommand{\andreapic}{
  \begin{tikzpicture}[oriented WD, bb min width =.7cm, bby=1.6ex, bbx=.7cm,bb port length=0pt] 
    \node[bb port sep=.8, bb={2}{1}, bb name=$\Sigma$] (Sigma1) {};
    \node[bb port sep=1.6,bb min width=4.3em, bb={2}{3}, below right=-2.5 and 1.5 of Sigma1.south east, bb name=Chassis] (Chassis) {};
    \node[bb port sep=.9,bb min width=4.3em, bb={2}{4}, below right=-3 and 2 of Chassis_out2, bb name=Motor] (Motor) {};
  	\node[bb port sep=.8, bb={2}{1}, right=1.5 of Motor.south east, bb name=$\Sigma$] (Sigma2) {};
    \node[bb={0}{0}, fit={($(Sigma1.north west)+(-.5,1)$) (Chassis) (Motor) ($(Sigma2.east)+(.25,0)$)}] (Y) {};
    \coordinate (Y_in1) at (Sigma1_in2-|Y.west);
    \coordinate (Y_in2) at (Chassis_in2-|Y.west);
    \coordinate (Y_out1) at (Motor_out2-|Y.east);
    \coordinate (Y_out2) at (Motor_out3-|Y.east);
    \coordinate (Y_out3) at (Sigma2_out1-|Y.east);
    \draw[ar] (Y_in1) to node[above=0pt, font=\tiny] {} (Sigma1_in2);
    \draw[ar] (Y_in2) to node[above=0pt, font=\tiny] {} (Chassis_in2);
    \draw[ar] (Sigma1_out1) to node[above=0pt, font=\tiny] {Weight} (Chassis_in1);
    \draw[ar] (Chassis_out1) to node[above=0pt, font=\tiny] {Torque} (Motor_in1);
    \draw[ar] (Chassis_out2) to node[above=0pt, font=\tiny] {Speed} (Motor_in2);
    \draw[ar] let \p1=(Motor.south west), \p2=(Motor.south east), \n1={\y1-\bby},\n2=\bbportlen in
    	(Chassis_out3) to node[above=0pt, font=\tiny] {Cost} (\x1-\n2,\n1) -- (\x2+\n2,\n1) to (Sigma2_in2);
    \draw[ar] (Motor_out2) to node[above=0pt, font=\tiny, pos=.8] {} (Y_out1);
    \draw[ar] (Motor_out4) to node[above=0pt, font=\tiny] {Cost} (Sigma2_in1);
    \draw[ar] (Motor_out3) to node[below=0pt, font=\tiny, pos=.8] {} (Y_out2);
    \draw[ar] (Sigma2_out1) to node[below=0pt, font=\tiny] {} (Y_out3); 
    \draw[ar] let \p1=(Motor.north east), \p2=(Sigma1.north west), \n1={\y2+\bby},\n2=\bbportlen in
    	(Motor_out1) to[in=0] node[right=0pt, font=\tiny] {Motor Weight} (\x1+\n2,\n1) -- (\x2-\n2,\n1) to[out=180] (Sigma1_in1);
  	\draw[label]
  		node[left=.3 of Y_in1] {\footnotesize Extra payload}
  		node[left=.3 of Y_in2] {\footnotesize Velocity}
  		node[right=.3 of Y_out1] {\footnotesize Voltage}
  		node[right=.3 of Y_out2] {\footnotesize Current}
  		node[right=.3 of Y_out3] {\footnotesize Cost \$}
%  		node[above right=.2 and .3 of Chassis_out1] {\tiny Torque}
%  		node[above right=.2 and .3 of Chassis_out2] {\tiny Speed}
%  		node[above right=.2 and .3 of Chassis_out3] {\tiny Cost \$}
%  		node[above right=.3 and .3 of Motor_out1, align=center, font=\tiny] {Motor\\weight}
	;	
  \end{tikzpicture}
}


\newcommand{\ivlpos}{(0,\infty)}
\newcommand{\ivlnon}{[0,\infty)}
\newcommand{\rrpos}{\rr_{>0}}
\newcommand{\rrnon}{\rr_{\geq0}}
\newcommand{\rrnonbar}{\bar{\rr}_{\geq0}}
\newcommand{\K}{\cat{K}}

\newcommand{\Kdag}[1][0]{\K\ldag}

\newcommand{\flr}[2][\blank]{\floor{#1}_{#2}}
\newcommand{\clg}[2][\blank]{\ceil{#1}_{#2}}
\newcommand{\tick}{\shortmid}
\newcommand{\ttick}{\shortmid\shortmid}
\newcommand{\dotminus}{\mathbin{\dot{-}}}
\renewcommand{\d}[1]{\;d #1}
\newcommand{\varint}{\textstyle\int}
\newcommand{\intzero}[1][\blank]{\varint_0^{#1}}
\newcommand{\xid}[1][\blank]{(#1)}


% Aaron extras



%Paulo extras








% ---- Changeable document parameters ---- %

\linespread{1.1}
\allowdisplaybreaks
\setsecnumdepth{section}
\settocdepth{section}
\setlength{\parindent}{15pt}

\settocdepth{chapter}

%--------------- Document ---------------%
\begin{document}

\title{Continuity and stability by co-design}

\author{
  Aaron Ames \and  \and 
  David I. Spivak \and\and 
  Paulo Tabuada}

\date{\vspace{-.2in}}

\maketitle

\begin{abstract}
\end{abstract}

%\renewcommand\cftchapteraftersnumb{\normalfont}
\renewcommand\cftbeforechapterskip{2pt plus 1pt}

%\begin{KeepFromToc}
%\tableofcontents
%\end{KeepFromToc}



\chapter{Introduction}
\label{chap.intro}

Lyapunov's notion of stability-by-comparison serves as the mathematical foundation of the control-theoretic perspective on dynamical systems. Category theory offers rich theory in which to account for any well-defined sort of comparison, so it is a natural fit.

In \cite{censi2015mathematical}, Andrea Censi used category theory to explain a theory of \emph{feasibility}, as exemplified in the practice of collaborative engineering design problems. For example, the feasibility of a robot---consisting of a chassis, a motor, and a battery---was considered in terms of traced monoidal categories. In their introductory book  \cite{fong2019seven} originally titled \emph{Seven Sketches in Compositionality}, Fong and Spivak offered a pedagogical account of feasibility theory 
\begin{equation}\label{eqn.andreadavidbrendan}
	\andreapic
\end{equation}
in terms of \emph{boolean profunctors}. 

In this paper, we use the mathematics behind co-design---the (Cartesian) monoidal equipment of \emph{boolean-enriched categories}---to recast Lyapunov's approach to stability. Though concise, the paper will be self-contained for those acquainted with
\begin{itemize}
	\item the basic theory of metric spaces: continuity and uniform continuity, and 
	\item the ``big-3'' of category theory: categories, functors, and natural transformations.
\end{itemize}
Though we will not include further pictures as in \eqref{eqn.andreadavidbrendan}, we will show how boolean profunctor theory also accounts for comparative stability, as we next briefly describe.

\paragraph{Basic analysis}

Continuity and uniform continuity **

\subsection{Natural Stability}

Lyapunov gave what turns out to be a beautiful categorical account of stability, though he did not know category theory. What he did know was that comparing dynamical systems---especially to the simplest ones---would offer a way to account for intuitions about energy in mathematical terms.

\dnote{(Expert Fill:) background on stability, comparing systems, and energy}


\paragraph{Class $\K$ functions in analysis.}

Ones intuitions about the world can be cast in the language of the nonnegative reals, $\ivlnon$, which is a highly structured object. Its additive commutative monoid structure $(\ivlnon,0,+)$ induces a poset structure $t_1\leq t_2$, defined by. **

\subsection{Simple systems and rounding: from $\K$ to $\Kdag$}

Rounding up and down: as the monad and comonad given by the adjoint endofunctors on $\rrnon$.

\subsection{Lyapunov in terms of boolean profunctors}

% https://q.uiver.app/#q=WzAsNCxbMCwwLCJYIl0sWzAsMSwiUiJdLFsxLDEsIlIiXSxbMSwwLCJZIl0sWzMsMiwiRyIsMix7ImN1cnZlIjoxfV0sWzAsMSwiRiIsMix7ImN1cnZlIjoxfV0sWzEsMiwiIiwyLHsic3R5bGUiOnsiYm9keSI6eyJuYW1lIjoiYmFycmVkIn19fV0sWzAsMSwiRiciLDAseyJjdXJ2ZSI6LTF9XSxbMywyLCJHJyIsMCx7ImN1cnZlIjotMX1dXQ==
\[
\begin{tikzcd}[row sep=1cm, column sep=2.7cm]
	X 
		\ar[dd, bend left=20pt, "F'"]
		\ar[dd, bend right=20pt, "F"']
		\ar[r, bend left=20pt, "\shortmid"{marking}", "~F'\leq_R G'", ""' name=C1]
		\ar[r, bend right=20pt, "\shortmid"{marking}", "F\leq_R G"' name=C2', "" name=C2]
	& 
	Y 
		\ar[dd, bend left=20pt, "G'"]
		\ar[dd, bend right=20pt, "G"'] 
	\\~\\
	R 
		\ar[r, "\shortmid\shortmid"{marking}", "\leq_R"' below=2pt, "" name=R]
	&
	R
	\ar[from=C1, to=C2, Rightarrow, ]
	\ar[from=C1, to=R, right, phantom, "\color{gray}\footnotesize~\tn{cart}"]
	\ar[from=C2, to=R, right, phantom, 
		"\color{gray}
		\footnotesize
		\!
		\text{cart}"
		]
\end{tikzcd}
\qquad
\parbox{1.3in}{\raggedright\textit{In the case of uniform continuity:}}
\;\;
\begin{tikzcd}[row sep=1cm, column sep=2.7cm]
	{(0,\infty)} 
		\ar[dd, bend left=20pt, hook, ""]
		\ar[dd, bend right=20pt, "\Delta"']
		\ar[r, bend left=20pt, "\shortmid"{marking}", "{\Delta(\epsilon)\geq d_M(m_1,m_2)}", ""' name=C1]
		\ar[r, bend right=20pt, "\shortmid"{marking}", "{\epsilon\geq d_N(fm_1,fm_2)}"' name=C2', "" name=C2]
	& 
	M\times M 
		\ar[dd, bend left=20pt, "d_M"]
		\ar[dd, bend right=20pt, "d_N\circ(f\times f)"'] 
	\\~\\
	\ivlnon 
		\ar[r, "\shortmid\shortmid"{marking}", "\geq"' below=2pt, "" name=R]
	&
	{[0,\infty)}
	\ar[from=C1, to=C2, Rightarrow, "\fun{ucont?}_f"]
	\ar[from=C1, to=R, right, phantom, pos=.65,
		"\color{gray}
		\footnotesize
		~
		\tn{cart}"]
	\ar[from=C2, to=R, right, phantom, pos=.65,
		"\color{gray}
		\footnotesize
		\!
		\text{cart}"
		]
\end{tikzcd}
\]
This 2-arrow labeled ``$\text{ucont?}_f$" asks the question:
\[
\Delta(\epsilon)\geq d_M(m_1,m_2)\xRightarrow{\;\;?\;\;}\epsilon\geq d_N\big(f(m_1),f(m_2)\big), \quad 
	\text{for all }\epsilon\in (0,\infty), (m_1,m_2)\in M\times M
\]
This is well-known to be the definition of uniform continuity. A function $\Delta\colon(0,\infty)\to(0,\infty)$ encodes the function which, for every $\epsilon>0$ chooses a $\delta\coloneqq\Delta(\epsilon)>0$. The implication labeled $\xRightarrow{?}$ says that for every two points $m_1,m_2$ whose distance is bounded by $\delta$, applying $f$ results in two points $f(m_1), f(m_2)$ whose distance is bounded by $\epsilon$. We will explain the diagrammatic approach in detail.

\subsection{Plan of the paper}

Continuity, 
uniform continuity, 
stability, and
class $\Kdag$-stability.

\subsection*{Acknowledgments}

MURI**

Spivak's work was additionally supported by the Air Force Office of Scientific Research under award number FA9550-23-1-0376.


\chapter{Basic analysis, stability theory, and profunctor theory}
\label{chap.stability_and_profunctors}


\section{Basic analysis}



\section{Basic stability theory}

\section{Basic profunctor theory}


From the perspective of co-design, the $\geq$-ordering tells us to think of a nonnegative real number $r\in[0,\infty]$ as something like a \emph{deviation}.


\begin{definition}
A \emph{Lawvere metric space} consists of a pair $(X,d)$, where $X$ is a set, $d\colon X\times X\to[0,\infty]$ is a function satisfying the following inequalities
\[
\begin{tikzpicture}[baseline=(P1.-10)]
	\node (P1) {
  \begin{tikzpicture}[oriented WD, bb small, bb port length=0, baseline=(zero)]
  	\node[circle, inner sep=1pt, draw] 				(zero) at (0,0) {};
  	\node[circle, inner sep=1pt, draw, fill] 	(term) at (1,0) {};
  	\draw (zero) -- +(-1,0);
  	\draw (term) -- +(1,0);
  \end{tikzpicture}
  };
  \node[right=of P1] (P2){
   \begin{tikzpicture}[oriented WD, bb small, bb port length=0, baseline=(d)]
  	\node[bb={1}{2}] (d) {$d$};
  	\draw (d_in1) --+(-1,0);
		\node[circle, inner sep=1pt, draw, fill] at ($(d.east)+(.5, 0)$) (dot) {};
		\draw (d_out1) to (dot);
		\draw (d_out2) to (dot);
		\draw (dot) -- +(1,0);
  \end{tikzpicture}
  };
  \node at ($(P1.east)!.5!(P2.west)$) {$\geq$};
\end{tikzpicture}
  \qqand
\begin{tikzpicture}[baseline=(P1.-10)]
	\node (P1) {
  \begin{tikzpicture}[oriented WD, bb small, bb port length=0, baseline=(zero)]
  	\node[circle, inner sep=1pt, draw] (sum) at (0,0) {};
		\node[circle, inner sep=1pt, draw, fill, right=3 of sum] (dot) {};
		\coordinate (helper1) at ($(sum)!.5!(dot)$);
		\coordinate (helper2) at ($(dot)+(1,0)$);
		\node[bb={1}{2}, above=of helper1] (d1) {$d$};
		\node[bb={1}{2}, below=of helper1] (d2) {$d$};
		\draw (sum) 		-- +(-1,0);
		\draw (sum) 		to (d1_in1);
		\draw (sum) 		to (d2_in1);
		\draw (d1_out1) -- (d1_out1-|helper2);
		\draw (d1_out2) to (dot);
		\draw (d2_out1) to (dot);
		\draw (dot) 		-- (dot-|helper2);
		\draw (d2_out2) -- (d2_out2-|helper2);
  \end{tikzpicture}
  };
  \node[right=of P1] (P2){
   \begin{tikzpicture}[oriented WD, bb small, bb port length=0, baseline=(d)]
  	\node[bb={1}{2}] (d) {$d$};
		\node[circle, inner sep=1pt, draw, fill, right=1 of d] (dot) {};
		\coordinate (helper1) at ($(d_out1)+(2, 1)$);
		\coordinate (helper2) at ($(d_out2)+(2, -1)$);
  	\draw (d_in1) --+(-1,0);
		\draw (d_out1) to (helper1);
		\draw (d_out2) to (helper2);
		\draw (dot) -- (dot-|helper1);
  \end{tikzpicture}
  };
  \node at ($(P1.east)!.5!(P2.west)$) {$\geq$};
\end{tikzpicture}  
\]
which express $0\geq d(x,x)$ and $d(x,y)+d(y,z)\geq d(x,z)$.
\end{definition}


\chapter{Continuity by co-design (boolean profunctors)}
\label{}

Let $(M,d_M)$ and $(N,d_N)$ be metric spaces. A function $f\colon M\to N$ is uniformly continuous iff there exists a function $\Delta\colon(0,\infty)\to(0,\infty)$ such that for all $\epsilon>0$ and all $(m_1,m_2)\in M\times M$, the following implication holds:
\[
\Delta(\epsilon)\geq d_M(m_1,m_2)\implies
\epsilon\geq d_N(f(m_1),f(m_2)).
\]

Thus, given $\Delta$ one forms the diagram left; it establishes uniform continuity if there exists a 2-morphism between the associated cartesian profunctors, as shown right:
\[
\begin{tikzcd}[row sep=1cm, column sep=3.2cm]
	{(0,\infty)} 
		\ar[dd, bend left=40pt, hook, ""]
		\ar[d, "\Delta?"']
%		\ar[r, bend left=20pt, "\shortmid"{marking}", "{\Delta(\epsilon)\geq d_M(m_1,m_2)}", ""' name=C1]
%		\ar[r, bend right=20pt, "\shortmid"{marking}", "{\epsilon\geq d_N(fm_1,fm_2)}"' name=C2', "" name=C2]
%		\ar[from=C1, to=C2, Rightarrow, "\fun{ucont?}_f"]
	& 
	M\times M 
		\ar[dd, bend left=40pt, "d_M"]
		\ar[d, "f\times f"'] 
	\\
	{(0,\infty)}
		\ar[d, hook]&
	N\times N
		\ar[d, "d_N"']\\
	\ivlnon 
		\ar[r, "\shortmid\shortmid"{marking}", "\geq"' below=2pt, "" name=R]
	&
	{[0,\infty)}
%	\ar[from=C1, to=R, right, phantom, pos=.65,
%		"\color{gray}
%		\footnotesize
%		~
%		\tn{cart}"]
%	\ar[from=C2, to=R, right, phantom, pos=.65,
%		"\color{gray}
%		\footnotesize
%		\!
%		\text{cart}"
%		]
\end{tikzcd}
\hspace{.6in}
\begin{tikzcd}[row sep=1cm, column sep=3.2cm]
	{(0,\infty)} 
		\ar[dd, bend left=40pt, hook, ""]
		\ar[d, "\Delta"']
		\ar[r, bend left=20pt, "\shortmid"{marking}", "{\Delta(\epsilon)\geq d_M(m_1,m_2)}", ""' name=C1]
		\ar[r, bend right=20pt, "\shortmid"{marking}", "{\epsilon\geq d_N(fm_1,fm_2)}"' name=C2', "" name=C2]
		\ar[from=C1, to=C2, Rightarrow, "\fun{ucont?}_f"]
	& 
	M\times M 
		\ar[dd, bend left=40pt, "d_M"]
		\ar[d, "f\times f"'] 
	\\
	{(0,\infty)}
		\ar[d, hook]&
	N\times N
		\ar[d, "d_N"']\\
	\ivlnon 
		\ar[r, "\shortmid\shortmid"{marking}", "\geq"' below=2pt, "" name=R]
	&
	{[0,\infty)}
	\ar[from=C1, to=R, right, phantom, pos=.65,
		"\color{gray}
		\footnotesize
		~
		\tn{cart}"]
	\ar[from=C2, to=R, right, phantom, pos=.65,
		"\color{gray}
		\footnotesize
		\!
		\text{cart}"
		]
\end{tikzcd}
\]

Similarly, we obtain continuity at a point. A function $f\colon M\to N$ is continuous at some $m_0\in M$ if there exists a function $\Delta\colon (0,\infty)\to(0,\infty)$ such that for all $\epsilon>0$ and all $m\in M$, the following implication holds:
\[
\Delta(\epsilon)\geq d_M(m_0,m)\implies
\epsilon\geq d_N(f(m_0),f(m)).
\]
We again obtain this by asking about the existence of a 2-cell between the cartesian morphisms, as shown
\[
\begin{tikzcd}[row sep=1cm, column sep=4cm]
	{(0,\infty)} 
		\ar[dd, bend left=40pt, hook, ""]
		\ar[d, "\Delta"']
		\ar[r, bend left=20pt, "\shortmid"{marking}", "{\Delta(\epsilon)\geq d_M(m_0,m)}", ""' name=C1]
		\ar[r, bend right=20pt, "\shortmid"{marking}", "{\epsilon\geq d_N(fm_0,fm)}"' name=C2', "" name=C2]
		\ar[from=C1, to=C2, Rightarrow, "\fun{cont?}_{f,m_0}"]
	& 
	M 
		\ar[dd, bend left=40pt, "{d_M(m_0,\blank)}"]
		\ar[d, "f"'] 
	\\
	{(0,\infty)}
		\ar[d, hook]&
	N
		\ar[d, "{d_M(f(m_0),\blank)}"']\\
	\ivlnon 
		\ar[r, "\shortmid\shortmid"{marking}", "\geq"' below=2pt, "" name=R]
	&
	{[0,\infty)}
	\ar[from=C1, to=R, right, phantom, pos=.65,
		"\color{gray}
		\footnotesize
		~
		\tn{cart}"]
	\ar[from=C2, to=R, right, phantom, pos=.65,
		"\color{gray}
		\footnotesize
		\!
		\text{cart}"
		]
\end{tikzcd}
\]

We now use the theory of equipments to show that the composite of uniformly continuous functions is uniformly continuous. The same can be done for continuity in place of uniform continuity.

\begin{proposition}
The identity on $M$ is uniformly continuous, and if $f\colon M\to N$ and $g\colon N\to O$ are uniformly continuous then so is $g\circ f$.
\end{proposition}
\begin{proof}
The function $f=\id_M$ is uniformly continuous because then one can take $\Delta=\id_{(0,\infty)}$ and the two maps are equal. 

Suppose that $f$ and $g$ are uniformly continuous.**
\end{proof}


\chapter{Properties of nonnegative reals}
\label{}

\section{Algebraic structure}

The poset $(\rrnon,\geq)$ or $([0,\infty),\geq)$ of nonnegative reals has a great deal of structure. They are \emph{semirings}---also called \emph{rigs} (rings without \emph{n}egatives)---meaning that we have operations $(\rrnon,0,+,1,*)$ with the usual properties. We also have $N$-ary operations
\[
	\sup_{n\in N}\colon\rrnon^N\to\rrnon
	\qqand
	\inf_{n\in N}\colon\rrnon^N\to\rrnon	
\]
which can be denoted by $\max$ and $\min$ when the arity $N$ is finite. For example,
\[
\max(x,y)\coloneqq
\begin{cases}
	x&\tn{ if }x\geq y\\
	y&\tn{ if }y\geq x
\end{cases}
\qqand
\min(x,y)\coloneqq
\begin{cases}
	x&\tn{ if }y\geq x\\
	y&\tn{ if }x\geq y
\end{cases}
\]
All of these structures and coherence properties also exist for the larger rig $[0,\infty]$ in place of $\rrnon$.

For any $y\in\rrnon$, shifting by $y$ has a right adjoint, defined as follows:
\[
x+y\geq z\ifff x\geq z\dotminus y
\where
z\dotminus y=\max(z-y,0)
\]
This defines a closure for the monoidal structure $(\rrnon,0,+)$.%
\footnote{This closure extends to a closure for $(\rrnonbar,0,+)$, putting $z\dotminus\infty\coloneqq 0$ for all $z\in[0,\infty]$.}

For any $y\in(0,\infty)$, scaling by $y$ also has a right adjoint, defined as follows:
\begin{equation}\label{eqn.scaling_adjoints}
x*y\geq z\ifff x\geq \frac{z}{y}
\end{equation}
This is not quite a closure on $(\rrnon,1,*)$, because it is not defined for $y=0$.%
\footnote{The existing adjoints extend to $\rrnonbar$, by putting $\frac{\infty}{y}=\infty$ for all $y\in(0,\infty)$ and $\frac{z}{\infty}\coloneqq 0$ for all $z\in[0,\infty]$. This is forced because for any $\epsilon>0$ one has $\epsilon*\infty\geq z$, so $\epsilon\geq\frac{z}{\infty}$. This in turn forces $0*\infty=\infty$. A rig is \emph{absorptive} if $0*x=0$ for all $x$; so if one wants to have the operation $\frac{\blank}{\infty}$, one must lose the absorptive property.}
For $y\in(0,\infty)$ and $x,z\in[0,\infty]$, the inequalities in \eqref{eqn.scaling_adjoints} become equalities
\[x*y = z\iff x = \frac{z}{y}.\]

\paragraph{Endofunctors of $\rrnon$.}

The poset of order-preserving maps $\cat{J}\coloneqq\{f\colon\rrnon\to\rrnon\}$, ordered pointwise
\[
f\geq g\ifff[\quad] f(r)\geq g(r),\quad \forall r\in\rrnon,
\]
is even richer. It has almost all of the structure that $\rrnon$ does: $(\cat{J},\leq,0,+,1,*,\sup,\inf)$. In particular the injection
\[
  \const\colon\rrnon\to\cat{J}
  \hi[,]{\qquad}{\quad}{given by}
  \const[r](s)=r
\]
preserves all of that structure. We sometimes denote $\const[r]$ simply by $r$, e.g.\ this explains the notation $0$ and $1$ as elements of $\cat{J}$ above.

The category $\cat{J}$ has a great deal of additional structure as well. First, it has a monoidal structure
\[
	\xid\in\cat{J}
	\qqand
	\circ:\cat{J}\times\cat{J}\to\cat{J}
\]
where we use $\xid$ for the identity functor $\xid[r]=r$ on $\rrnon$. For example, we have
\[\xid[f]=f=f\xid.\]
The composition operation respects the order on $\cat{J}$: if $f\geq f'$ and $g\geq g'$, then $f\circ g\geq f'\circ g'$. 

%The set $\cat{J}'$ of subidentical elements
%\[
%\cat{J}'\coloneqq\{f\in\cat{J}\mid f\leq\id\}
%\]
%form a subcategory with other structures $(\cat{J}',0,*)$.

For any $y\in[0,\infty)$, we have a map $(\dotminus y)\colon\cat{J}\to\cat{J}$ and for any $y\in(0,\infty]$, we have a map $\frac{\blank}{y}\colon\cat{J}\to\cat{J}$; these are given by
\[
	(f\dotminus y)(x)\coloneqq f(x)\dotminus y
	\qqand
	\textstyle\frac{f}{y}(x)\displaystyle\coloneqq \frac{f(x)}{y}.
\]

There is also a functor $\intzero\colon\cat{J}\to\cat{J}$ given by
\[
f\mapsto t\mapsto\int_0^t f(\tau) \d \tau.
\]
In particular, $\intzero f$ is an order preserving function $\rrnon\to\rrnon$ for each $f\in\cat{J}$, and $\intzero$ preserves the order between functions themselves:
\[
t_1\leq 
t_2
\implies 
\int_0^{t_1}f(\tau)\d\tau\leq
\int_0^{t_2}f(\tau)\d\tau
\qqand
f_1\leq 
f_2
\implies
\intzero f_1\leq
\intzero f_2.
\]
It is linear, i.e.\ strong monoidal with respect to $(0,+)$
\[
\intzero 0=0
\qqand
\intzero f+\intzero g=\intzero (f+g)
\]
Composing with $\const$ we obtain a functor $\intzero\colon(\rrnon,1,*)\to(\cat{J},\xid,\circ)$, called  \emph{accumulation} which is strong monoidal:
\[
\intzero 1=\xid
\qqand
(\intzero r)\circ(\intzero s)=\intzero(r*s)
\]

Accumulation also preserves $\inf$ and $\sup$. It preserves $\dotminus y$ for any $y\in[0,\infty)$ and $\frac{\blank}{y}$ for any $y\in(0,\infty]$.

Much of the above also works for the category $\cat{J}^+$ of endofunctors $([0,\infty],\geq)\to([0,\infty],\geq)$ in place of $\cat{J}$. 

For any $r\in[0,\infty]$ there is a functor $\rest{r}\colon\cat{J}^+\to\cat{J}^+$ given by
\[
\rest[f]{r}(x)=
	\begin{cases}
		f(x)&\tn{ if }r\geq x\\
		\infty&\tn{ if } x>r
	\end{cases}
\]
For example, $\chi\coloneqq\rest[0]{0}$ is extreme: $\chi(0)=0$ and $\chi(r)=\infty$ for all $r>0$. 
We can now define a monoidal structure $(\cat{J}^+,\chi,\star)$, where we refer to $\chi$ as \emph{extremis} and $\star$ as the \emph{join}, defined as follows:
\[
(f\star g)(x)=
\begin{cases}
	f(x)&\tn{ if }r> x\\
	f(r)+g(x\dotminus r)&\tn{ if }x\geq r
\end{cases}
\;\where
r\coloneqq\inf_{f(x)=\infty}x.
\]
One checks that $\chi\star f=f=f\star\chi$ for all $f\in\cat{J}^+$, as well as associativity.

\begin{example}
Using the $(\cat{J}^+,\chi,\star)$ monoidal structure can obtain the following:
\begin{equation}\label{eqn.lowersemicont}
\rest[f]{r}\star g=
\begin{tikzpicture}[baseline=(cop)]
	\draw[->, very thin] (0, 0) -- (7, 0);
	\draw[->, very thin] (0, 0) -- (0, 2);
	\node (sm) at (2,0) {$\shortmid$};
	\node[below=1pt] at (sm) {$r$};
	\begin{scope}[blue!70!black]
  	\node[circle, draw, inner sep=1pt, fill] at (2,.8) (ccl) {};
  	\draw (0, 0) to[out=5, in=200] (ccl);
  	\node[circle, draw, inner sep=1pt] at (2,1) (cop) {};
  	\draw[->] (cop) to[out=20, in=195] (6.8, 1.7);
	\end{scope}
\end{tikzpicture}
\end{equation}
for $r\in[0,\infty]$ and particular continuous functions $f,g$. We will see this sort of piecewise-defined functions as part of a larger category---the limiting notion---in the next section.

\end{example}

\section{Adjoint endofunctors of the nonnegative reals}

Adjoint functors $(\rrnon,\geq)\to(\rrnon,\geq)$ can be thought of in terms of graph transposes.
\[
\begin{tikzpicture}[x=.7cm, y=.5cm]
	\draw [xstep=1, ystep=1, very thin, densely dotted] (0,0) grid (5,5);
	\draw [very thin, ->] (0,0) -- (5.5, 0) node[below left] {$x$};
	\draw [very thin, ->] (0,0) -- (0, 5.5) node[below left] {$y$};
	\begin{scope}[blue!70!black, thick]
  	\node[circle, draw, inner sep=1pt, fill] at (1,0) (ccl) {};
  	\node[circle, draw, inner sep=1pt] 			 at (1,2) (cop) {};
		\draw (0,0) -- (ccl);
		\draw (cop) -- (2,3) -- (4,3) to[out=10, in=200] (5,5);
	\end{scope}
	\foreach \i in {0,..., 4} {
		\node[left, font=\tiny] at (0, \i) {\i};
		\node[below, font=\tiny] at (\i, 0) {\i};
	}
	\node[above] at (2.5, 5) {$y=f(x)$};
\end{tikzpicture}
\hspace{.6in}
\begin{tikzpicture}[x=.7cm, y=.5cm]
	\draw [xstep=1, ystep=1, very thin, densely dotted] (0,0) grid (5,5);
	\draw [very thin, ->] (0,0) -- (5.5, 0) node[below left] {$x$};
	\draw [very thin, ->] (0,0) -- (0, 5.5) node[below left] {$y$};
	\begin{scope}[blue!70!black, thick]
  	\node[circle, draw, inner sep=1pt, fill] at (4,3) (ccl) {};
  	\node[circle, draw, inner sep=1pt] 			 at (2,3) (cop) {};
		\draw (1,0) -- (1,2) -- (cop);
		\draw (ccl) to[out=10, in=200] (5,5);
	\end{scope}
	\foreach \i in {0,..., 4} {
		\node[left, font=\tiny] at (0, \i) {\i};
		\node[below, font=\tiny] at (\i, 0) {\i};
	}
	\node[right] at (5, 2.5) {$x=f\ldag(y)$};
\end{tikzpicture}
\]
For example, the graph to the right is that of a function $y=f(x)$ for some nondecreasing function (functor); for example $0=f(1)$ and $2=f(2)$. That to the left is its left adjoint $x=f\ldag(y)$, here it is drawn with $x$ as a function of $y$, namely we have drawn $x=f\ldag(y)$ on the same grid; for example $1=f\ldag(0)$ and $1=f\ldag(2)$.



Throughout this section we assume that $\alpha\colon R\to R'$ is nondecreasing, but each of $R$ and $R'$ might be the poset carried by any of the following sets:
\begin{equation}\label{eqn.RR}
	R,R'\text{~are any of:~}\qquad
	[0,\infty]\hh[\qquad]{or}
	[0,\infty)\hh[\qquad]{or}
	(0,\infty]\hh[\qquad]{or}
	(0,\infty),
\end{equation}
equipped with the $\geq$ ordering. For the time being we fix $R$ and $R'$ but remain agnostic about which of the posets from \eqref{eqn.RR} they may be. Then $\alpha$ may or may not have a left adjoint, meaning a nondecreasing function $\alpha\ldag\colon R'\to R$ such that
\begin{equation}\label{eqn.unit_counit}
  \epsilon\geq\alpha(\alpha\ldag(\epsilon))
  \qqand
  \alpha\ldag(\alpha(\delta))\geq\delta.
\end{equation}
If one exists then as such it will be unique, and we write
\[
\adjrpos{R}{\alpha}{\alpha\ldag}{R'}{\geq}.
\]


\begin{example}
With $R,R'$ as in \eqref{eqn.RR}, any monotonic bijection $\alpha\colon R\to R'$ is a right (also a left) adjoint. The constant-$0$ function $[0,\infty]\to R'$ given by $\alpha(r)=0$, is a right adjoint; its left adjoint is $\alpha\ldag(r')=\infty$. The ceiling function $[0,\infty)\to[0,\infty)$ is a right adjoint and the floor function is its left adjoint
\[
	\alpha(r)=\ceil{r}=\inf_{\{n:\nn\mid n\geq r\}}n
	\qqand
	\alpha\ldag(r')=\floor{r}=\sup_{\{n:\nn\mid r'\geq n\}}n
\]
\end{example}

\begin{lemma}\label{lemma.radj_sups}
For every nondecreasing map $f\colon R\to R'$, and subset $S\ss R$, we have
\begin{equation}\label{eqn.sup_lax}
	f(\sup_{s\in S}s)
	\geq
	\sup_{s\in S}f(s)
\end{equation}
If $f$ is a right adjoint, then the converse also holds: $f$ preserves all sups.
\end{lemma}
\begin{proof}
	\Cref{eqn.sup_lax} follows from the fact that $f(\sup_{s\in S}s)\geq\alpha (s_0)$ for each $s_0\in S$. If $f$ is a right adjoint, with left adjoint $f\ldag$, then it follows from \cref{eqn.unit_counit,eqn.sup_lax} that:
	\[
	\sup_s f(s)\geq
	f\circ f\ldag\Big(\sup_s f(s)\Big)\geq
	f\Big(\sup_s f\ldag\circ f(s)\Big)\geq
	f\Big(\sup_s s\Big).
	\qedhere
	\]
\end{proof}


\begin{lemma}\label{lemma.sup_formula}
Suppose given a nondecreasing function $\alpha\colon R\to R'$. If $\alpha$ has a left adjoint $\alpha\ldag$, then $\alpha\ldag$ will satisfy the following formula for any $\epsilon\in R'$:
\begin{equation}\label{eqn.alpha_left}
  \alpha\ldag(\epsilon)=\sup_{\{\delta\in R\mid \epsilon\geq\alpha(\delta)\}}\delta.
\end{equation}
Conversely, if $\alpha$ preserves sups and \eqref{eqn.alpha_left} is defined for all $\epsilon\in R'$, then $\alpha\ldag$ is left adjoint to $\alpha$.
\end{lemma}
\begin{proof}
Suppose given \eqref{eqn.unit_counit}. To show $\alpha\ldag(\epsilon)\geq\sup_{\{\delta\mid \epsilon\geq\alpha(\delta)\}}\delta$, it suffices to show that for all $\delta\in R'$, if $\epsilon\geq\alpha(\delta)$ then $\alpha\ldag(\epsilon)\geq\delta$. By \eqref{eqn.unit_counit}, if $\epsilon\geq\alpha(\delta)$ then $\alpha\ldag(\epsilon)\geq\alpha\ldag(\alpha(\delta))\geq\delta$. 

By \cref{lemma.radj_sups}, $\alpha$ preserves sups. To show 
$
\sup_{\{\delta\mid \epsilon\geq\alpha(\delta)\}}\delta
\geq
\alpha\ldag(\epsilon)
$, we again use \eqref{eqn.unit_counit}:
\[
\left(\sup_{\{\delta\mid \epsilon\geq\alpha(\delta)\}}\delta \right)\geq
\left(\sup_{\{\epsilon'\mid \epsilon\geq\alpha(\alpha\ldag(\epsilon'))\}}\alpha\ldag(\epsilon')\right)\geq\alpha\ldag(\epsilon)
\]

For the converse, suppose that $\alpha$ preserves sups and \eqref{eqn.alpha_left} is defined for all $\epsilon\in R'$. One checks easily that \eqref{eqn.unit_counit} holds for any $\epsilon_0\in R'$ and $\delta_0\in R$:
\[
	\epsilon_0
	\geq
  \sup_{\{\delta\in R\mid \epsilon_0\geq\alpha(\delta)\}}\alpha(\delta)
  =\alpha(\alpha\ldag(\epsilon_0))
  \qqand
  \alpha\ldag(\alpha(\delta_0))=
  \sup_{\{\delta\in R\mid \alpha(\delta_0)\geq\alpha(\delta)\}}\delta
  \geq\delta_0
\]

\end{proof}

%\begin{example}
%Here is a table of some maps that do or do not have left adjoints:
%\[
%\begin{array}{lll||c|c}
%	\text{Domain}&\text{Codomain}&\text{Monotonic $\alpha(\delta)$}&\text{Left adj.\ }\alpha\ldag(\epsilon)?&\text{Reason not}\\\hline
%	\blank,\blank&\blank,\blank&\delta+k \quad(k>0)&\text{N/A}&\delta\geq\alpha\ldag(\delta)+1\geq1\\
%	\blank,\blank&\blank,\blank&\delta*k \quad(k>0)&\frac{\epsilon}{k}&\text{N/A}\\
%	\blank,\infty]&[0,\blank&0&\infty&\text{N/A}\\
%	\blank,\blank&[0,\blank&\flr[\delta]{}&\text{N/A}&\alpha\text{ doesn't prsv. sups}\\
%	\blank,\blank&[0,\blank&\clg[\delta]{}&\flr[\epsilon]{}&\text{N/A}
%\end{array}
%\]
%\end{example}

\paragraph{Lower semicontinuity}

The function shown in \cref{eqn.lowersemicont} is lower semicontinuous.%
\footnote{Lower semicontinuous functions do not need to be piecewise continuous. For example, consider the function $f\colon\rrnon\to\rrnon$ that is $0$ on $[0, \frac{1}{2})$, $\frac{1}{2}$ on $[\frac{1}{2}, \frac{3}{4})$, etc. up to $1$, and then is $1$ on $[1,\infty)$. It is lower semicontinuous but has infinitely many discontinuities in any open interval containing $1$.}  
 These turn out to be quite related to adjoint functors. A nondecreasing function $f\colon R\to R'$ is lower semicontinuous iff, for every $r'\in R'$, the preimage $f\inv[0,r']$ is closed, or in other words $f(\sup_{r\mid r'\geq f(r)}r)=r'$.

\begin{lemma}\label{lemma.sups_lower_semi}
If $\alpha\colon R\to R'$ is nondecreasing and preserves sups then it is lower-semicontinuous. 

Conversely, if $\alpha$ is lower semicontinuous and either $0\not\in R$ or $0\in R$ and $\alpha(0)=0$, then it preserves sups.
\end{lemma}
\begin{proof}
Recall that $\alpha$ is lower semicontinuous if for each $r'\in R'$, the set $S_{r'}\coloneqq\{r:R\mid r'\geq\alpha(r)\}$ is closed. Since $\alpha$ is nondecreasing, $S_{r'}$ is an interval and the question is whether it contains its upper endpoint. If $\alpha$ preserves sups, then $S_{r'}$ contains its upper endpoint, i.e.\ $\big(\sup_{r\in S_{r'}}r\big)\in S_{r'}$, for each $r'\in R'$. 

Conversely, suppose $A_{\leq r'}$ contains its upper endpoint for each $r'\in R'$. Let $S\ss R$ be a subset; we consider it case by case. If $S$ is empty, and $0\not\in R$ then this sup-preservation condition is vacuous; if $0\in R$ then $\sup_{s\in S}s=0=\sup_{s\in S}\alpha(s)$, so this sup is preserved. If $S$ is nonempty and finite, then the sup is the max and its preservation reduces to the fact that $\alpha$ is nondecreasing. 

Finally, suppose $S$ is infinite and let $r_S\coloneqq\sup_{s\in S}s$ and $r'\coloneqq\sup_{s\in S}\alpha(s)$. We clearly have $S\ss S_{r'}$ and $\big(\sup_{r\in S_{r'}}r\big)\geq r_S$. Then $\alpha$ preserves this sup (i.e.\ $r'=\alpha(r_S)$) because: $\alpha(r_S)\geq r'$ holds by \eqref{eqn.sup_lax}, and $r'=\alpha\big(\sup_{r\in S_{r'}}r\big)\geq\alpha(r_S)$ holds by lower semicontinuity.
\end{proof}

\begin{theorem}\label{thm.omnibus}
Suppose $R,R'$ are as above and that $\alpha\colon R\to R'$ is nondecreasing. Define the following named properties:
\begin{description}[leftmargin=\parindent,labelindent=\parindent]
	\item[LS:] $\alpha$ is lower semicontinuous.
	\item[PZ:] $\alpha$ preserves zero, i.e.\ $0=\alpha(0)$.
	\item[UA:] $\alpha$ is unbounded above, i.e.\ $\infty=\lim_{\delta\to\infty}\alpha(\delta)$.
	\item[UB:] $\alpha$ is unbounded below, i.e.\ $0=\lim_{\delta\to0}\alpha(\delta)$.
\end{description}
where implicitly, if $0\in R$ and $0\not\in R'$ then (PZ) cannot hold. We can characterize right adjoints as follows:
\begin{itemize}
	\item If $R=[0,\infty]$, then $\alpha$ is a right adjoint iff LS, PZ.
	\item If $R=[0,\infty)$, then $\alpha$ is a right adjoint iff LS, PZ, UA.
%	\item If $R=(0,\infty]$, and $0\in R'$ then $\alpha$ is a right adjoint iff LS, UB.
%	\item If $R=(0,\infty]$, and $0\not\in R'$ then $\alpha$ is a right adjoint iff LS, UB.
	\item If $R=(0,\infty]$, then $\alpha$ is a right adjoint iff LS, UB.
%	\item If $R=(0,\infty)$, and $0\in R'$ then $\alpha$ is a right adjoint iff LS, UA, UB.
%	\item If $R=(0,\infty)$, and $0\not\in R'$ then $\alpha$ is a right adjoint iff LS, UA, UB.
	\item If $R=(0,\infty)$, then $\alpha$ is a right adjoint iff LS, UB, UA.
\end{itemize}
\end{theorem}
\begin{proof}
We first show the $\Rightarrow$ direction. By \cref{lemma.radj_sups,lemma.sups_lower_semi}, if $\alpha$ is a right adjoint, then it preserves sups and is lower semicontinuous (LS holds). If $0\in R$ then it is the empty sup, so $\alpha(0)=0$ (PZ holds). If $\infty\not\in R$ then for all $\epsilon\in R'$ there exists $\delta\in R$ with $\alpha(\delta)>\epsilon$ by \eqref{eqn.alpha_left}, so $\alpha$ is unbounded above (UA holds). If $0\not\in R$ and $0\in R'$ then by \eqref{eqn.alpha_left} there exists $\delta\in R$ with $0\geq\alpha(\delta)$, i.e.\ $\alpha(\delta)=0$; since $\alpha$ is nondecreasing, we have $0=\lim_{\delta\to 0}\alpha(\delta)$ (UB holds). If $0\not\in R$ and $0\not\in R'$ then by \eqref{eqn.alpha_left} for all $\epsilon>0$ there exists $\delta>0$ with $\epsilon\geq\alpha(\delta)$ (UB holds).

By \cref{lemma.sups_lower_semi}, in each case the named properties are enough to show that $\alpha$ preserves sups. They are also chosen to be enough that \eqref{eqn.alpha_left} is well-defined for all $\epsilon\in R'$. Hence, by \cref{lemma.sup_formula}, in each case the named properties imply that $\alpha$ is a right adjoint.
\end{proof}

%The composite of $\alpha$ and $\alpha\ldag$ in either order are reminiscent of ``rounding'' up or down.
%
%\begin{lemma}[Monads/comonads as rounding up/down]
%Suppose $\alpha\ldag$ is left adjoint to $\alpha$:
%\[
%\adjpos{R'}{\alpha\ldag}{\alpha}{R}{\geq}.
%\]
%For any $\delta\in R$ and $\epsilon\in R'$, define
%\[
%	\clg[\delta]{\alpha}\coloneqq\alpha\ldag(\alpha(\delta))\quad\in R
%	\qqand
%	\flr[\epsilon]{\alpha}\coloneqq\alpha(\alpha\ldag(\epsilon))\quad\in R'
%\]
% Then for any $r\in R$ and $r'\in R'$ we have:
%% \[
%% 	\epsilon\geq r'\geq\flr[\epsilon]{\alpha}
%%	\implies
%%	\flr[r']{\alpha}
%%	=
%% 	\flr[\epsilon]{\alpha}
%%	.\]
%%	Similarly, for any $\delta\in R$, let $\clg[\delta]{\alpha}\coloneqq\alpha\ldag(\alpha(\delta))\in R$. Then for any $r\in R$ we have the implication:
%% \[
%% 	\clg[\delta]{\alpha}\geq r\geq\delta
%%	\implies
%% 	\clg[\delta]{\alpha}
%%	=
%%	\clg[r']{\alpha}
%%	.\]	
%\[
%  \clg[{\clg[r]{\alpha}}]{\alpha}=\clg[r]{\alpha}.
%  \qqand
%  \flr[{\flr[r']{\alpha}}]{\alpha}=\flr[r']{\alpha}
%\]
%In particular, we have implications
% \[
% 	\clg[\delta]{\alpha}\geq r\geq\delta
%	\implies
% 	\clg[\delta]{\alpha}
%	=
%	\clg[r]{\alpha}	
%	\qqand
% 	\epsilon\geq r'\geq\flr[\epsilon]{\alpha}
%	\implies
%	\flr[r']{\alpha}
%	=
% 	\flr[\epsilon]{\alpha}
%	.\]
%\end{lemma}
%\begin{proof}
%For the first claim we calculate:
%\[
%  \alpha\ldag\circ\alpha\geq
%  \alpha\ldag\circ\alpha\circ\alpha\ldag\circ\alpha\geq
%  \alpha\ldag\circ\alpha
%  \qqand
%  \alpha\circ\alpha\ldag\geq
%  \alpha\circ\alpha\ldag\circ\alpha\circ\alpha\ldag\geq
%  \alpha\circ\alpha\ldag
%\]
%For the second, we apply $\clg{\alpha}$ to $\clg[\delta]{\alpha}\geq r\geq\delta$ and use the first claim to conclude $\clg[\delta]{\alpha}\geq \clg[r]{\alpha}\geq\clg[\delta]{\alpha}$. Similarly, we apply $\flr{\alpha}$ to $\epsilon\geq r'\geq\flr[\epsilon]{\alpha}$ and use the first claim to conclude $\flr[\epsilon]{\alpha}\geq \flr[r']{\alpha}\geq\flr[\epsilon]{\alpha}$.
%\end{proof}

%\begin{proposition}\label{prop.cts_adjoint_close}
%Let $R\coloneqq [0,\infty]$, and suppose $\alpha\colon R\to R'$ is continuous. Then $\alpha$ has a left adjoint iff $0\in R'$ and $\alpha(0)=0$.
%\end{proposition}
%\begin{proof}
%	If $\alpha$ has a left adjoint $\alpha\ldag$, then $0\geq\alpha(\alpha\ldag(0))\geq\alpha(0)$. For the converse, suppose $0\in R'$ and $\alpha(0)=0$. Then $\alpha$ preserves all sups: it preserves the empty sup, and all other sups follow from monotonicity and continuity. Defining $\alpha\ldag$ as in \eqref{eqn.alpha_left}, one checks that both inequalities in \eqref{eqn.unit_counit} hold.
%\end{proof}
%
%\begin{proposition}\label{prop.cts_v1}
%Let $R\coloneqq [0,\infty)$, and suppose $\alpha\colon R\to R'$ is continuous. Then $\alpha$ has a left adjoint iff $0\in R'$, $\alpha(0)=0$, and $\alpha$ is surjective.
%\end{proposition}
%\begin{proof}
%	The proof of \cref{prop.cts_adjoint_close} goes through until one attempts to define $\alpha\ldag$ as in \eqref{eqn.alpha_left}. When $\alpha$ is surjective then the proof proceeds as before, so surjectivity is sufficient. However, if $\alpha$ is not surjective, i.e.\ if there exists $\epsilon$ such that $\epsilon\geq\alpha(\delta)$ for all $\delta$, then $\alpha\ldag(\epsilon)$ needs to be $\infty$, but it is not in the domain. Thus surjectivity is necessary.
%\end{proof}
%
%\begin{proposition}\label{prop.cts_v2}
%Suppose $0\in R,R'$, and suppose $\alpha\colon R\to R'$ is continuous. If it has a left adjoint $\alpha\ldag$, then it satisfies
%\begin{equation}\label{eqn.adj_restriction}
%	\alpha\ldag(\epsilon)=0\implies
%	\epsilon=0.
%\end{equation}
%In other words, letting $i\colon R-\{0\}\inj R$ be the inclusion, then under the above conditions there exists a dashed arrow making the following diagram commute:
%\begin{equation}\label{eqn.adj_restriction_diagram}
%\begin{tikzcd}
%	R-\{0\}\ar[r, "i"]\ar[d, dashed, "\Delta_\alpha"']&
%	R\ar[d, "\alpha\ldag"]\\
%	R'-\{0\}\ar[r, "i"']&
%	R'
%\end{tikzcd}
%\end{equation}
%%In this case, $\Delta_\alpha$ is itself a left adjoint iff $\alpha\inv(0)=\{0\}$.
%\end{proposition}
%\begin{proof}
%First note that we satisfy either the hypotheses of \cref{prop.cts_v1} or \cref{prop.cts_v2}, so $\alpha(0)=0$. Now suppose $\alpha\ldag(\epsilon)=0$. By \cref{lemma.sup_formula}, we have $\sup_{\{\delta|\epsilon\geq\alpha(\delta)\}}\delta=0$, so $\{\delta\mid\epsilon\geq\alpha(\delta)\}=\{0\}$. Since $\alpha$ is
%continuous, $\epsilon=0$. This proves the first claim, which is equivalent to the existence of the dashed arrow as in \eqref{eqn.adj_restriction}.
%%
%%For the last claim, if $\Delta_\alpha$ has a right adjoint
%\end{proof}


\begin{corollary}\label{cor.restrict_adjoint}
If $\alpha'\colon(0,\infty)\to (0,\infty)$ is a right adjoint, then it is the restriction of a right adjoint $\alpha\colon[0,\infty)\to [0,\infty)$.
\end{corollary}
\begin{proof}
Suppose $\alpha'$ is a right adjoint. By \cref{thm.omnibus}, it satisfies LS, UA, and UB. Define $\alpha\colon[0,\infty)\to [0,\infty)$ by $\alpha(0)\coloneqq 0$ and $\alpha(r)\coloneqq\alpha'(r)$ for $r>0$. Then it preserves 0 (PZ) and satisfies UA because $\alpha'$ does. For any $r'\in[0,\infty)$, we have
\[\alpha\inv[0,r']=\{0\}\cup(\alpha')\inv(0,r'],\]
which is closed in $[0,\infty)$. Hence $\alpha$ satisfies LS and is a right adjoint, again by \cref{thm.omnibus}.
\end{proof}

\begin{definition}
We say a monotone map $\alpha\colon[0,\infty)\to [0,\infty)$ is \emph{class $\Kdag$} if it restricts to a right adjoint $\alpha'\colon(0,\infty)\to(0,\infty)$.
\end{definition}

By \cref{cor.restrict_adjoint}, a map $\alpha$ is class $\Kdag$ iff it fits into a diagram
\[
\begin{tikzcd}[column sep=large]
	{(0,\infty)}\ar[r, shift left=4pt, "\alpha'"]\ar[d, hook, "i"']&
	{(0,\infty)}\ar[l, shift left=4pt, "{(\alpha')\ldag}"]\ar[d, hook, "i"]\\
	{[0,\infty)}\ar[r, shift left=4pt, "\alpha"]&
	{[0,\infty)}\ar[l, shift left=4pt, "\alpha\ldag"]
\end{tikzcd}
\]
where the commutation is $i\circ\alpha'=\alpha\circ i$ and $i\circ(\alpha')\ldag=\alpha\ldag\circ i$.

\begin{corollary}
For a monotone function $\alpha\colon[0,\infty)\to[0,\infty)$, the following are equivalent:
\begin{enumerate}
	\item $\alpha$ is class $\Kdag$.
	\item $\alpha$ restricts to a right adjoint $\alpha\big|_{(0,\infty)}\colon(0,\infty)\to (0,\infty)$.
	\item $\alpha$ is a right adjoint and it restricts to a right adjoint $\alpha\big|_{(0,\infty)}\colon(0,\infty)\to (0,\infty)$.
	\item $\alpha$ is lower semicontinuous, continuous at 0, and unbounded above, and $\alpha(0)=0$.
\end{enumerate}
\end{corollary}
\begin{proof}

\end{proof}


\chapter{Classical and $\Kdag$-stability by co-design}
\label{}

\begin{definition}[Dynamical system]
Let $(T,0,+)$ be a monoid. A \emph{$T$-dynamical system} consists of a pair $(X,\Phi)$, where $X$ is a set and $\Phi\colon T\times X\to X$ is a $T$-action, meaning it satisfies
\[
  \Phi(0,x)=x
  \qqand
  \Phi(t_1+t_2,x)=\Phi(t_1,\Phi(t_2,x)).
\]
\end{definition}


\begin{definition}[Classical stability]
For a set $X$ and norm $\|\blank\|\colon X\to\rrnon$, a dynamical system $\Phi\colon T\times X\to X$ is \emph{classically stable} if: for every $\epsilon>0$ there exists a $\delta>0$ such that for all $t\in T$ and $x\in X$, the following implication holds:
\[
	\delta\geq\|x\|
	\qimplies
	\epsilon\geq\|\Phi(t,x)\|
	\qedhere
\]
\end{definition}
Assuming the axiom of choice, we can translate into the constructive version.

\begin{definition}[Constructive classical stability]
For $X$, $\|\blank\|\colon X\to\rrnon$, and $\Phi\colon T\times X\to X$ as above, a \emph{classical stability structure} consists of a function $\Delta\colon(0,\infty)\to(0,\infty)$ such that for all $\epsilon>0$, $t\in T$, and $x\in X$, the following implication holds:
\[
	\Delta(\epsilon)\geq\|x\|
	\qimplies
	\epsilon\geq\|\Phi(t,x)\|
	\qedhere
\]
\end{definition}

We can phrase the latter as a comparison of co-design problems (boolean-profunctors) having the form $((0,\infty),\geq)\tickar T\times X$. The right-hand side is considered as a discrete poset. Such profunctors are thus simply functions
\[
P,Q\colon (0,\infty)\times T\times X\to\{\false,\true\}
\]
such that if $P(\epsilon, t, x)$ holds and $\epsilon'\geq\epsilon$ then $P(\epsilon', t, x)$ also; similarly for $Q$. The two profunctors are
\[
	P(\epsilon, t, x)\coloneqq (\Delta(\epsilon)\geq^?\|x\|)	
  \qqand
	Q(\epsilon, t, x)\coloneqq (\epsilon\geq^?\|\Phi(t,x)\|)
\]
The comparison question can be written concisely as left or verbosely as right:
\[
\begin{tikzcd}[column sep=2cm]
	\rrpos
		\ar[r, bend left=18pt, tick, "P", ""' name=P]
		\ar[r, bend right=18pt, tick, "Q"', "" name=Q]&
	T\times X
		\ar[from=P, to=P|-Q, Rightarrow, "?"]
\end{tikzcd}
\qqor
\forall(\epsilon,t,x)\in\rrpos\times T\times X,\quad
P(\epsilon, t, x)\Rightarrow^? Q(\epsilon, t, x)
\]

In fact, we can construct these as follows:
\[
\begin{tikzcd}[column sep=65pt]
	{(0,\infty)}
		\ar[d, "\Delta"']
		\ar[dd, bend left=40pt, hook, "i"]
		\ar[r, bend left=18pt, tick, "\Delta(\epsilon)\geq\|x\|", ""' name=P]
		\ar[r, bend right=18pt, tick, "{\epsilon\geq\|\Phi(t,x)\|}"', "" name=Q]
	&
	T\times X\ar[d, shift right=4pt, "\pi"']\ar[d, shift left=4pt, "\Phi"]\\
	{0,\infty)}\ar[d, hook, "i"']\ar[r, phantom, gray, "\text{cart}"]&
	X\ar[d, "\|\blank\|"]\\
	{[0,\infty)}\ar[r, tick, "\geq"']&
	{[0,\infty)}
	\ar[from=P, to=Q-|P, Rightarrow, "?"]
\end{tikzcd}
\]

\begin{proposition}\label{prop.lyapunov}
Let $X$ be a manifold and $\Phi\colon\rrnon\times X\to X$ a smooth action of $(\rrnon,0,+)$, and let $f\colon X\to TX$ be the vector field given by
\[f(x)\coloneqq \frac{\partial\Phi}{\partial t}\big|_{(0,x)}=\lim_{t\to 0}\frac{\Phi(t,x)}{t}.\] 
For a function $V\colon X\to\rrnon$, the vector field formula $\vec{0}\geq TV\circ f$ holds iff we have $V(x)\geq V(\Phi(t,x))$ for all $t,x$. That is, the square to the right laxly commutes iff the square to the left laxly commutes:
\[
\begin{tikzcd}
	\rrnon\times X\ar[r, "\Phi"]\ar[d, "\pi"']&
	X\ar[r, "f"]\ar[d, "V"']&
	TX\ar[d, "TV"]\\
	X\ar[r, "V"']&
	\rrnon\ar[r, "\vec{0}"']\ar[ul, phantom, "\Rightarrow"]&
	T\rrnon\ar[ul, phantom, "\Rightarrow"]
\end{tikzcd}
\]
\end{proposition}
\begin{proof}
For all $x,t$, we have $V(x)\geq V(\Phi(t,x))$ iff $0\geq V(\Phi(t,x))-V(x)$. So if $V(x)\geq V(\Phi(t,x))$ then, noting that $x=\Phi(0,x)$, dividing both sides by $t>0$ and taking the limit, we have
\[
0\geq\lim_{t\to 0}\frac{(V\circ\Phi)(t,x))-(V\circ\Phi)(0,x)}{t}=TV\circ f.
\]
In the other direction, one uses the comparison lemma. \dnote{Improve this.}
\end{proof}

\begin{definition}[Lyapunov function]
Given a smooth action $\Phi\colon\rrnon\times X\to X$, a function $V\colon X\to\rrnon$ satisfying the two equivalent conditions in \cref{prop.lyapunov} is called a \emph{Lyapunov function} for $\Phi$.
\end{definition}

\paragraph{Class $\Kdag$ stability implies classical stability}

\[
\begin{tikzcd}
	{(0,\infty)}
		\ar[d, hook, "i"']\ar[r, "\ttick"{marking, name=A1}, "\geq" below=1pt]
		\ar[rr, bend left=25pt, "i\geq \|\Phi\|", ""' name=D1]
	&
	|[alias=D2]|{(0,\infty)}
		\ar[d, "\Delta"]
		\ar[r, "\tick"{marking}, "i\circ\Delta\geq\|\pi\|"']
	&[15pt]
	T\times X
		\ar[d, "\pi"]\ar[ddr, bend left=20pt, "\Phi"]
	\\
	{[0,\infty)}
		\ar[dd, shift right=5pt, bend right=50pt, equal, "" name=C1]\ar[d, "\alpha\ldag"']
	&
	{(0,\infty)}
		\ar[d, hook, "i"]
	&
	X
		\ar[l, phantom, "\text{cart}"]
		\ar[d, "\|\blank\|"]
	\\
	|[alias=C2]|{[0,\infty)}
		\ar[r, "\ttick"{marking, name=A2}, "\geq" below=1pt]
		\ar[d, "\alpha"']
	&
	|[alias=B1]|{[0,\infty)}
		\ar[r, "\ttick"{marking}, "\geq" below=1pt]
	&
	{[0,\infty)}
		\ar[d, "\alpha"]\ar[r, Rightarrow, yshift=6pt, shorten=1pt]
	&[-10pt]
	X
		\ar[dl, bend left, "\|\blank\|"]
	\\
	{[0,\infty)}\ar[rr, "\ttick"{marking, name=B2}, "\geq" below=1pt]
	&&
	{[0,\infty)}
		\ar[from=A1, to=A2, Rightarrow, shorten=25pt]
		\ar[from=B1.south-|B2, to=B2, Rightarrow, shorten=3pt, "\alpha"]
		\ar[from=C1, to=C2, Rightarrow, shorten=-2pt]
		\ar[from=D1, to=D1|-D2, pos=.3, phantom, "\Uparrow?"]
\end{tikzcd}
\]

Suppose given class-$\Kdag$-stability, which is the right-hand 2-cell, where $\alpha$ is class-$\Kdag$. Then we form the lefthand 2-cell as the unit of the adjunction; the bottom 2-cell is the identity on $\alpha$. The map $\Delta$ is $(\alpha')\ldag$ from \cref{cor.restrict_adjoint} and the 2-cell is the identity on $\alpha\ldag\circ i=i\circ\Delta$. 

We obtain classical stability because for all $\epsilon\in(0,\infty)$ and $(t,e)\in T\times X$, and suppressing the inclusions $i$, we have:
\[
  \Delta(\epsilon)\geq \|x\|
  \quad\implies\quad
	\epsilon\geq\|\Phi(t,x)\|.
\]


\chapter{Miscellaneous and Future work}

Kan extensions bounding positive definite functions.





\end{document}
